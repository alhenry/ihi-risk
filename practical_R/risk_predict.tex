% Options for packages loaded elsewhere
\PassOptionsToPackage{unicode}{hyperref}
\PassOptionsToPackage{hyphens}{url}
%
\documentclass[
]{article}
\usepackage{amsmath,amssymb}
\usepackage{lmodern}
\usepackage{iftex}
\ifPDFTeX
  \usepackage[T1]{fontenc}
  \usepackage[utf8]{inputenc}
  \usepackage{textcomp} % provide euro and other symbols
\else % if luatex or xetex
  \usepackage{unicode-math}
  \defaultfontfeatures{Scale=MatchLowercase}
  \defaultfontfeatures[\rmfamily]{Ligatures=TeX,Scale=1}
\fi
% Use upquote if available, for straight quotes in verbatim environments
\IfFileExists{upquote.sty}{\usepackage{upquote}}{}
\IfFileExists{microtype.sty}{% use microtype if available
  \usepackage[]{microtype}
  \UseMicrotypeSet[protrusion]{basicmath} % disable protrusion for tt fonts
}{}
\makeatletter
\@ifundefined{KOMAClassName}{% if non-KOMA class
  \IfFileExists{parskip.sty}{%
    \usepackage{parskip}
  }{% else
    \setlength{\parindent}{0pt}
    \setlength{\parskip}{6pt plus 2pt minus 1pt}}
}{% if KOMA class
  \KOMAoptions{parskip=half}}
\makeatother
\usepackage{xcolor}
\usepackage[margin=1in]{geometry}
\usepackage{color}
\usepackage{fancyvrb}
\newcommand{\VerbBar}{|}
\newcommand{\VERB}{\Verb[commandchars=\\\{\}]}
\DefineVerbatimEnvironment{Highlighting}{Verbatim}{commandchars=\\\{\}}
% Add ',fontsize=\small' for more characters per line
\usepackage{framed}
\definecolor{shadecolor}{RGB}{248,248,248}
\newenvironment{Shaded}{\begin{snugshade}}{\end{snugshade}}
\newcommand{\AlertTok}[1]{\textcolor[rgb]{0.94,0.16,0.16}{#1}}
\newcommand{\AnnotationTok}[1]{\textcolor[rgb]{0.56,0.35,0.01}{\textbf{\textit{#1}}}}
\newcommand{\AttributeTok}[1]{\textcolor[rgb]{0.77,0.63,0.00}{#1}}
\newcommand{\BaseNTok}[1]{\textcolor[rgb]{0.00,0.00,0.81}{#1}}
\newcommand{\BuiltInTok}[1]{#1}
\newcommand{\CharTok}[1]{\textcolor[rgb]{0.31,0.60,0.02}{#1}}
\newcommand{\CommentTok}[1]{\textcolor[rgb]{0.56,0.35,0.01}{\textit{#1}}}
\newcommand{\CommentVarTok}[1]{\textcolor[rgb]{0.56,0.35,0.01}{\textbf{\textit{#1}}}}
\newcommand{\ConstantTok}[1]{\textcolor[rgb]{0.00,0.00,0.00}{#1}}
\newcommand{\ControlFlowTok}[1]{\textcolor[rgb]{0.13,0.29,0.53}{\textbf{#1}}}
\newcommand{\DataTypeTok}[1]{\textcolor[rgb]{0.13,0.29,0.53}{#1}}
\newcommand{\DecValTok}[1]{\textcolor[rgb]{0.00,0.00,0.81}{#1}}
\newcommand{\DocumentationTok}[1]{\textcolor[rgb]{0.56,0.35,0.01}{\textbf{\textit{#1}}}}
\newcommand{\ErrorTok}[1]{\textcolor[rgb]{0.64,0.00,0.00}{\textbf{#1}}}
\newcommand{\ExtensionTok}[1]{#1}
\newcommand{\FloatTok}[1]{\textcolor[rgb]{0.00,0.00,0.81}{#1}}
\newcommand{\FunctionTok}[1]{\textcolor[rgb]{0.00,0.00,0.00}{#1}}
\newcommand{\ImportTok}[1]{#1}
\newcommand{\InformationTok}[1]{\textcolor[rgb]{0.56,0.35,0.01}{\textbf{\textit{#1}}}}
\newcommand{\KeywordTok}[1]{\textcolor[rgb]{0.13,0.29,0.53}{\textbf{#1}}}
\newcommand{\NormalTok}[1]{#1}
\newcommand{\OperatorTok}[1]{\textcolor[rgb]{0.81,0.36,0.00}{\textbf{#1}}}
\newcommand{\OtherTok}[1]{\textcolor[rgb]{0.56,0.35,0.01}{#1}}
\newcommand{\PreprocessorTok}[1]{\textcolor[rgb]{0.56,0.35,0.01}{\textit{#1}}}
\newcommand{\RegionMarkerTok}[1]{#1}
\newcommand{\SpecialCharTok}[1]{\textcolor[rgb]{0.00,0.00,0.00}{#1}}
\newcommand{\SpecialStringTok}[1]{\textcolor[rgb]{0.31,0.60,0.02}{#1}}
\newcommand{\StringTok}[1]{\textcolor[rgb]{0.31,0.60,0.02}{#1}}
\newcommand{\VariableTok}[1]{\textcolor[rgb]{0.00,0.00,0.00}{#1}}
\newcommand{\VerbatimStringTok}[1]{\textcolor[rgb]{0.31,0.60,0.02}{#1}}
\newcommand{\WarningTok}[1]{\textcolor[rgb]{0.56,0.35,0.01}{\textbf{\textit{#1}}}}
\usepackage{longtable,booktabs,array}
\usepackage{calc} % for calculating minipage widths
% Correct order of tables after \paragraph or \subparagraph
\usepackage{etoolbox}
\makeatletter
\patchcmd\longtable{\par}{\if@noskipsec\mbox{}\fi\par}{}{}
\makeatother
% Allow footnotes in longtable head/foot
\IfFileExists{footnotehyper.sty}{\usepackage{footnotehyper}}{\usepackage{footnote}}
\makesavenoteenv{longtable}
\usepackage{graphicx}
\makeatletter
\def\maxwidth{\ifdim\Gin@nat@width>\linewidth\linewidth\else\Gin@nat@width\fi}
\def\maxheight{\ifdim\Gin@nat@height>\textheight\textheight\else\Gin@nat@height\fi}
\makeatother
% Scale images if necessary, so that they will not overflow the page
% margins by default, and it is still possible to overwrite the defaults
% using explicit options in \includegraphics[width, height, ...]{}
\setkeys{Gin}{width=\maxwidth,height=\maxheight,keepaspectratio}
% Set default figure placement to htbp
\makeatletter
\def\fps@figure{htbp}
\makeatother
\setlength{\emergencystretch}{3em} % prevent overfull lines
\providecommand{\tightlist}{%
  \setlength{\itemsep}{0pt}\setlength{\parskip}{0pt}}
\setcounter{secnumdepth}{-\maxdimen} % remove section numbering
\ifLuaTeX
  \usepackage{selnolig}  % disable illegal ligatures
\fi
\IfFileExists{bookmark.sty}{\usepackage{bookmark}}{\usepackage{hyperref}}
\IfFileExists{xurl.sty}{\usepackage{xurl}}{} % add URL line breaks if available
\urlstyle{same} % disable monospaced font for URLs
\hypersetup{
  pdftitle={Practical: RISK PREDICTION},
  hidelinks,
  pdfcreator={LaTeX via pandoc}}

\title{Practical: RISK PREDICTION}
\usepackage{etoolbox}
\makeatletter
\providecommand{\subtitle}[1]{% add subtitle to \maketitle
  \apptocmd{\@title}{\par {\large #1 \par}}{}{}
}
\makeatother
\subtitle{Advanced Statistics for Records Research}
\author{}
\date{\vspace{-2.5em}}

\begin{document}
\maketitle

\hypertarget{research-question}{%
\subsection{Research question}\label{research-question}}

In this session, we will explore the dataset of 2000 participants we met
in the lecture, and fit a risk prediction model for death within 5
years, based on some simple patient characteristics.

\hypertarget{objectives}{%
\subsection{Objectives}\label{objectives}}

By the end of this practical, you should be able to:

\begin{enumerate}
\def\labelenumi{\arabic{enumi}.}
\item
  Fit a logistic model to create risk predictions.
\item
  Assess model discrimination by calculating the Area Under the Curve.
\item
  Assess model calibration by graphing observed and predicted risks.
\end{enumerate}

\hypertarget{dataset-and-analysis}{%
\subsection{Dataset and analysis}\label{dataset-and-analysis}}

For this practical we will use a (simulated) dataset called
\texttt{data\_predict}. This contains data for 2,000 patients, with
information on six variables.

\begin{longtable}[]{@{}ll@{}}
\toprule()
Variable & Description \\
\midrule()
\endhead
id & Unique patient ID \\
age & Age (years) \\
sbp & Systolic Blood Pressure (mm/Hg) \\
bmi & Body Max Index \(kg/m^2\) \\
sex & Female / Male \\
dead & Alive / Dead \\
\bottomrule()
\end{longtable}

\begin{Shaded}
\begin{Highlighting}[]
\CommentTok{\# Install Libraries}
\ControlFlowTok{if}\NormalTok{ (}\SpecialCharTok{!}\FunctionTok{require}\NormalTok{(pacman)) }\FunctionTok{install.packages}\NormalTok{(}\StringTok{"pacman"}\NormalTok{)}
\CommentTok{\#\textgreater{} Loading required package: pacman}
\NormalTok{pacman}\SpecialCharTok{::}\FunctionTok{p\_load}\NormalTok{(tidyr, dplyr, ggplot2, broom, here, gtsummary)}

\CommentTok{\# Load data}
\FunctionTok{load}\NormalTok{(here}\SpecialCharTok{::}\FunctionTok{here}\NormalTok{(}\StringTok{"data/data\_predict.rda"}\NormalTok{))}
\end{Highlighting}
\end{Shaded}

\hypertarget{data-exploration}{%
\subsection{Data exploration}\label{data-exploration}}

Have a look at the data, then try answering the following participants:
* How many participants died at the end of the follow-up? * What is the
proportion of female participants? * What is the mean, standard
deviation, minimum, and maximum age of these participants?

\begin{Shaded}
\begin{Highlighting}[]
\CommentTok{\# Base R}
\FunctionTok{summary}\NormalTok{(data\_predict)}
\CommentTok{\#\textgreater{}        id              age             sbp             bmi            sex      }
\CommentTok{\#\textgreater{}  Min.   :   1.0   Min.   :40.00   Min.   : 76.6   Min.   :15.50   Female: 978  }
\CommentTok{\#\textgreater{}  1st Qu.: 500.8   1st Qu.:51.00   1st Qu.:113.7   1st Qu.:23.20   Male  :1022  }
\CommentTok{\#\textgreater{}  Median :1000.5   Median :60.00   Median :120.6   Median :25.10                }
\CommentTok{\#\textgreater{}  Mean   :1000.5   Mean   :60.45   Mean   :120.3   Mean   :25.19                }
\CommentTok{\#\textgreater{}  3rd Qu.:1500.2   3rd Qu.:70.00   3rd Qu.:127.2   3rd Qu.:27.20                }
\CommentTok{\#\textgreater{}  Max.   :2000.0   Max.   :80.00   Max.   :152.2   Max.   :35.60                }
\CommentTok{\#\textgreater{}     dead     }
\CommentTok{\#\textgreater{}  Alive:1491  }
\CommentTok{\#\textgreater{}  Dead : 509  }
\CommentTok{\#\textgreater{}              }
\CommentTok{\#\textgreater{}              }
\CommentTok{\#\textgreater{}              }
\CommentTok{\#\textgreater{} }

\CommentTok{\# Tidyverse (More verbose but more control)}
\NormalTok{data\_predict }\SpecialCharTok{\%\textgreater{}\%} 
  \FunctionTok{group\_by}\NormalTok{(dead) }\SpecialCharTok{\%\textgreater{}\%} 
\NormalTok{  tally }\SpecialCharTok{\%\textgreater{}\%} 
  \FunctionTok{mutate}\NormalTok{(}\AttributeTok{percent =}\NormalTok{ n}\SpecialCharTok{/}\FunctionTok{sum}\NormalTok{(n)}\SpecialCharTok{*}\DecValTok{100}\NormalTok{)}
\CommentTok{\#\textgreater{} \# A tibble: 2 x 3}
\CommentTok{\#\textgreater{}   dead      n percent}
\CommentTok{\#\textgreater{}   \textless{}fct\textgreater{} \textless{}int\textgreater{}   \textless{}dbl\textgreater{}}
\CommentTok{\#\textgreater{} 1 Alive  1491    74.6}
\CommentTok{\#\textgreater{} 2 Dead    509    25.4}

\NormalTok{data\_predict }\SpecialCharTok{\%\textgreater{}\%} 
  \FunctionTok{group\_by}\NormalTok{(sex) }\SpecialCharTok{\%\textgreater{}\%} 
\NormalTok{  tally }\SpecialCharTok{\%\textgreater{}\%} 
  \FunctionTok{mutate}\NormalTok{(}\AttributeTok{percent =}\NormalTok{ n}\SpecialCharTok{/}\FunctionTok{sum}\NormalTok{(n)}\SpecialCharTok{*}\DecValTok{100}\NormalTok{)}
\CommentTok{\#\textgreater{} \# A tibble: 2 x 3}
\CommentTok{\#\textgreater{}   sex        n percent}
\CommentTok{\#\textgreater{}   \textless{}fct\textgreater{}  \textless{}int\textgreater{}   \textless{}dbl\textgreater{}}
\CommentTok{\#\textgreater{} 1 Female   978    48.9}
\CommentTok{\#\textgreater{} 2 Male    1022    51.1}

\NormalTok{data\_predict }\SpecialCharTok{\%\textgreater{}\%} 
  \CommentTok{\# filter out missing observation at any variable}
  \FunctionTok{filter\_all}\NormalTok{(}\FunctionTok{all\_vars}\NormalTok{(}\SpecialCharTok{!}\FunctionTok{is.na}\NormalTok{(.))) }\SpecialCharTok{\%\textgreater{}\%} 
  \FunctionTok{summarise}\NormalTok{(}\AttributeTok{n =} \FunctionTok{n}\NormalTok{(),}
            \AttributeTok{mean =} \FunctionTok{mean}\NormalTok{(age),}
            \AttributeTok{sd =} \FunctionTok{sd}\NormalTok{(age),}
            \AttributeTok{min =} \FunctionTok{min}\NormalTok{(age),}
            \AttributeTok{max =} \FunctionTok{max}\NormalTok{(age))}
\CommentTok{\#\textgreater{} \# A tibble: 1 x 5}
\CommentTok{\#\textgreater{}       n  mean    sd   min   max}
\CommentTok{\#\textgreater{}   \textless{}int\textgreater{} \textless{}dbl\textgreater{} \textless{}dbl\textgreater{} \textless{}dbl\textgreater{} \textless{}dbl\textgreater{}}
\CommentTok{\#\textgreater{} 1  2000  60.5  11.5    40    80}

\CommentTok{\# or with gtsummary }
\CommentTok{\# https://www.danieldsjoberg.com/gtsummary/index.html}
\FunctionTok{tbl\_summary}\NormalTok{(data\_predict)}
\CommentTok{\#\textgreater{} Table printed with \textasciigrave{}knitr::kable()\textasciigrave{}, not \{gt\}. Learn why at}
\CommentTok{\#\textgreater{} https://www.danieldsjoberg.com/gtsummary/articles/rmarkdown.html}
\CommentTok{\#\textgreater{} To suppress this message, include \textasciigrave{}message = FALSE\textasciigrave{} in code chunk header.}
\end{Highlighting}
\end{Shaded}

\begin{longtable}[]{@{}lc@{}}
\toprule()
\textbf{Characteristic} & \textbf{N = 2,000} \\
\midrule()
\endhead
id & 1,000 (501, 1,500) \\
age & 60 (51, 70) \\
sbp & 121 (114, 127) \\
bmi & 25.10 (23.20, 27.20) \\
sex & \\
Female & 978 (49\%) \\
Male & 1,022 (51\%) \\
event & \\
Alive & 1,491 (75\%) \\
Dead & 509 (25\%) \\
\bottomrule()
\end{longtable}

About 25\% of the 2,000 individuals die -- this is a high-risk
population. There is roughly 50\% males and 50\% females, aged 40-80.

\hypertarget{randomly-split-data-into-training-and-test-sets}{%
\subsubsection{Randomly split data into training and test
sets}\label{randomly-split-data-into-training-and-test-sets}}

\begin{Shaded}
\begin{Highlighting}[]
\FunctionTok{set.seed}\NormalTok{(}\DecValTok{777}\NormalTok{)}

\NormalTok{n\_total }\OtherTok{\textless{}{-}} \FunctionTok{nrow}\NormalTok{(data\_predict)}
\NormalTok{proportion\_train }\OtherTok{\textless{}{-}} \FloatTok{0.5}
\NormalTok{n\_train }\OtherTok{\textless{}{-}} \FunctionTok{floor}\NormalTok{(proportion\_train }\SpecialCharTok{*}\NormalTok{ n\_total)}

\NormalTok{sample\_train }\OtherTok{\textless{}{-}} \FunctionTok{sample}\NormalTok{(}\DecValTok{1}\SpecialCharTok{:}\NormalTok{n\_total, n\_train) }\SpecialCharTok{\%\textgreater{}\%}\NormalTok{ sort}
\NormalTok{sample\_test }\OtherTok{\textless{}{-}} \FunctionTok{which}\NormalTok{(}\SpecialCharTok{!}\NormalTok{(}\DecValTok{1}\SpecialCharTok{:}\NormalTok{n\_total }\SpecialCharTok{\%in\%}\NormalTok{ sample\_train))}

\NormalTok{data\_train }\OtherTok{\textless{}{-}}\NormalTok{ data\_predict[sample\_train,]}
\NormalTok{data\_test }\OtherTok{\textless{}{-}}\NormalTok{ data\_predict[sample\_test,]}
\end{Highlighting}
\end{Shaded}

\hypertarget{fit-model-in-training-data}{%
\subsubsection{Fit model in training
data}\label{fit-model-in-training-data}}

\begin{Shaded}
\begin{Highlighting}[]
\CommentTok{\# Fit logistic regression}
\NormalTok{model }\OtherTok{\textless{}{-}} \FunctionTok{glm}\NormalTok{(}\AttributeTok{formula =}\NormalTok{ dead }\SpecialCharTok{\textasciitilde{}}\NormalTok{ age }\SpecialCharTok{+}\NormalTok{ sex }\SpecialCharTok{+}\NormalTok{ sbp }\SpecialCharTok{+}\NormalTok{ bmi,}
             \AttributeTok{family =} \StringTok{"binomial"}\NormalTok{,}
             \AttributeTok{data =}\NormalTok{ data\_train)}

\CommentTok{\# View model output summary}
\FunctionTok{summary}\NormalTok{(model)}
\CommentTok{\#\textgreater{} }
\CommentTok{\#\textgreater{} Call:}
\CommentTok{\#\textgreater{} glm(formula = dead \textasciitilde{} age + sex + sbp + bmi, family = "binomial", }
\CommentTok{\#\textgreater{}     data = data\_train)}
\CommentTok{\#\textgreater{} }
\CommentTok{\#\textgreater{} Deviance Residuals: }
\CommentTok{\#\textgreater{}     Min       1Q   Median       3Q      Max  }
\CommentTok{\#\textgreater{} {-}1.9187  {-}0.7526  {-}0.4669   0.6399   2.7754  }
\CommentTok{\#\textgreater{} }
\CommentTok{\#\textgreater{} Coefficients:}
\CommentTok{\#\textgreater{}               Estimate Std. Error z value Pr(\textgreater{}|z|)    }
\CommentTok{\#\textgreater{} (Intercept) {-}13.297927   1.440507  {-}9.231  \textless{} 2e{-}16 ***}
\CommentTok{\#\textgreater{} age           0.086203   0.007921  10.883  \textless{} 2e{-}16 ***}
\CommentTok{\#\textgreater{} sexMale       0.155126   0.160836   0.965    0.335    }
\CommentTok{\#\textgreater{} sbp           0.052679   0.008465   6.223 4.87e{-}10 ***}
\CommentTok{\#\textgreater{} bmi           0.013306   0.027782   0.479    0.632    }
\CommentTok{\#\textgreater{} {-}{-}{-}}
\CommentTok{\#\textgreater{} Signif. codes:  0 \textquotesingle{}***\textquotesingle{} 0.001 \textquotesingle{}**\textquotesingle{} 0.01 \textquotesingle{}*\textquotesingle{} 0.05 \textquotesingle{}.\textquotesingle{} 0.1 \textquotesingle{} \textquotesingle{} 1}
\CommentTok{\#\textgreater{} }
\CommentTok{\#\textgreater{} (Dispersion parameter for binomial family taken to be 1)}
\CommentTok{\#\textgreater{} }
\CommentTok{\#\textgreater{}     Null deviance: 1126.86  on 999  degrees of freedom}
\CommentTok{\#\textgreater{} Residual deviance:  950.39  on 995  degrees of freedom}
\CommentTok{\#\textgreater{} AIC: 960.39}
\CommentTok{\#\textgreater{} }
\CommentTok{\#\textgreater{} Number of Fisher Scoring iterations: 5}

\CommentTok{\# Get odds ratio and 95\% CI}
\FunctionTok{cbind}\NormalTok{(}\AttributeTok{OR =} \FunctionTok{coef}\NormalTok{(model), }\FunctionTok{confint}\NormalTok{(model)) }\SpecialCharTok{\%\textgreater{}\%}\NormalTok{ exp}
\CommentTok{\#\textgreater{} Waiting for profiling to be done...}
\CommentTok{\#\textgreater{}                       OR        2.5 \%       97.5 \%}
\CommentTok{\#\textgreater{} (Intercept) 1.677968e{-}06 9.341761e{-}08 2.662586e{-}05}
\CommentTok{\#\textgreater{} age         1.090028e+00 1.073596e+00 1.107490e+00}
\CommentTok{\#\textgreater{} sexMale     1.167805e+00 8.524485e{-}01 1.602203e+00}
\CommentTok{\#\textgreater{} sbp         1.054091e+00 1.037001e+00 1.072023e+00}
\CommentTok{\#\textgreater{} bmi         1.013395e+00 9.596049e{-}01 1.070147e+00}

\CommentTok{\# tidy method}
\FunctionTok{tidy}\NormalTok{(model)}
\CommentTok{\#\textgreater{} \# A tibble: 5 x 5}
\CommentTok{\#\textgreater{}   term        estimate std.error statistic  p.value}
\CommentTok{\#\textgreater{}   \textless{}chr\textgreater{}          \textless{}dbl\textgreater{}     \textless{}dbl\textgreater{}     \textless{}dbl\textgreater{}    \textless{}dbl\textgreater{}}
\CommentTok{\#\textgreater{} 1 (Intercept) {-}13.3      1.44       {-}9.23  2.67e{-}20}
\CommentTok{\#\textgreater{} 2 age           0.0862   0.00792    10.9   1.39e{-}27}
\CommentTok{\#\textgreater{} 3 sexMale       0.155    0.161       0.965 3.35e{-} 1}
\CommentTok{\#\textgreater{} 4 sbp           0.0527   0.00846     6.22  4.87e{-}10}
\CommentTok{\#\textgreater{} 5 bmi           0.0133   0.0278      0.479 6.32e{-} 1}
\end{Highlighting}
\end{Shaded}

\hypertarget{predict-risk-based-on-trained-model}{%
\subsubsection{Predict risk based on trained
model}\label{predict-risk-based-on-trained-model}}

\hypertarget{on-training-dataset}{%
\paragraph{On training dataset}\label{on-training-dataset}}

\begin{Shaded}
\begin{Highlighting}[]
\CommentTok{\# update data\_train and data\_test with predicted probabilites}
\NormalTok{data\_train}\SpecialCharTok{$}\NormalTok{prob\_dead }\OtherTok{\textless{}{-}} 
  \FunctionTok{predict}\NormalTok{(model, }\AttributeTok{type =} \StringTok{"response"}\NormalTok{)}

\CommentTok{\# compare predicted risk }
\NormalTok{data\_train }\SpecialCharTok{\%\textgreater{}\%} 
  \FunctionTok{group\_by}\NormalTok{(dead) }\SpecialCharTok{\%\textgreater{}\%} 
  \FunctionTok{summarise}\NormalTok{(}\AttributeTok{n =} \FunctionTok{n}\NormalTok{ (),}
            \AttributeTok{mean =} \FunctionTok{mean}\NormalTok{(prob\_dead),}
            \AttributeTok{sd =} \FunctionTok{sd}\NormalTok{(prob\_dead),}
            \AttributeTok{min =} \FunctionTok{min}\NormalTok{(prob\_dead),}
            \AttributeTok{max =} \FunctionTok{max}\NormalTok{(prob\_dead)) }\SpecialCharTok{\%\textgreater{}\%} 
  \FunctionTok{pivot\_longer}\NormalTok{(}\SpecialCharTok{{-}}\NormalTok{dead) }\SpecialCharTok{\%\textgreater{}\%} 
  \FunctionTok{pivot\_wider}\NormalTok{(}\AttributeTok{id\_cols =}\NormalTok{ name, }\AttributeTok{names\_from =}\NormalTok{ dead)}
\CommentTok{\#\textgreater{} \# A tibble: 5 x 3}
\CommentTok{\#\textgreater{}   name     Alive     Dead}
\CommentTok{\#\textgreater{}   \textless{}chr\textgreater{}    \textless{}dbl\textgreater{}    \textless{}dbl\textgreater{}}
\CommentTok{\#\textgreater{} 1 n     749      251     }
\CommentTok{\#\textgreater{} 2 mean    0.207    0.383 }
\CommentTok{\#\textgreater{} 3 sd      0.155    0.189 }
\CommentTok{\#\textgreater{} 4 min     0.0134   0.0213}
\CommentTok{\#\textgreater{} 5 max     0.841    0.820}
\end{Highlighting}
\end{Shaded}

\hypertarget{on-test-dataset}{%
\paragraph{On test dataset}\label{on-test-dataset}}

\begin{Shaded}
\begin{Highlighting}[]
\NormalTok{data\_test}\SpecialCharTok{$}\NormalTok{prob\_dead }\OtherTok{\textless{}{-}}
  \FunctionTok{predict}\NormalTok{(model, }\AttributeTok{type =} \StringTok{"response"}\NormalTok{, }\AttributeTok{newdata =}\NormalTok{ data\_test)}

\NormalTok{data\_test }\SpecialCharTok{\%\textgreater{}\%} 
  \FunctionTok{group\_by}\NormalTok{(dead) }\SpecialCharTok{\%\textgreater{}\%} 
  \FunctionTok{summarise}\NormalTok{(}\AttributeTok{n =} \FunctionTok{n}\NormalTok{ (),}
            \AttributeTok{mean =} \FunctionTok{mean}\NormalTok{(prob\_dead),}
            \AttributeTok{sd =} \FunctionTok{sd}\NormalTok{(prob\_dead),}
            \AttributeTok{min =} \FunctionTok{min}\NormalTok{(prob\_dead),}
            \AttributeTok{max =} \FunctionTok{max}\NormalTok{(prob\_dead)) }\SpecialCharTok{\%\textgreater{}\%} 
  \FunctionTok{pivot\_longer}\NormalTok{(}\SpecialCharTok{{-}}\NormalTok{dead) }\SpecialCharTok{\%\textgreater{}\%} 
  \FunctionTok{pivot\_wider}\NormalTok{(}\AttributeTok{id\_cols =}\NormalTok{ name, }\AttributeTok{names\_from =}\NormalTok{ dead)}
\CommentTok{\#\textgreater{} \# A tibble: 5 x 3}
\CommentTok{\#\textgreater{}   name     Alive     Dead}
\CommentTok{\#\textgreater{}   \textless{}chr\textgreater{}    \textless{}dbl\textgreater{}    \textless{}dbl\textgreater{}}
\CommentTok{\#\textgreater{} 1 n     742      258     }
\CommentTok{\#\textgreater{} 2 mean    0.212    0.406 }
\CommentTok{\#\textgreater{} 3 sd      0.160    0.192 }
\CommentTok{\#\textgreater{} 4 min     0.0149   0.0256}
\CommentTok{\#\textgreater{} 5 max     0.850    0.844}
\end{Highlighting}
\end{Shaded}

\begin{Shaded}
\begin{Highlighting}[]
\CommentTok{\# Alternative solution (with broom::augment() )}
\NormalTok{alt\_data\_train }\OtherTok{\textless{}{-}}\NormalTok{ model }\SpecialCharTok{\%\textgreater{}\%}
  \CommentTok{\# augment creates new columns with some useful information from the model}
  \CommentTok{\# .fitted = predicted values}
  \FunctionTok{augment}\NormalTok{(}\AttributeTok{type.predict =} \StringTok{"response"}\NormalTok{)}

\NormalTok{alt\_data\_test }\OtherTok{\textless{}{-}}\NormalTok{ model }\SpecialCharTok{\%\textgreater{}\%}
  \FunctionTok{augment}\NormalTok{(}\AttributeTok{type.predict =} \StringTok{"response"}\NormalTok{, }\AttributeTok{newdata =}\NormalTok{ data\_test)}
\end{Highlighting}
\end{Shaded}

\hypertarget{validation}{%
\subsubsection{Validation}\label{validation}}

\hypertarget{roc}{%
\paragraph{ROC}\label{roc}}

In the training dataset, the ROC is 79\%. This means that a person who
did die has a 79\% probability of having a higher predicted risk (of
dying) than someone who did not. This shows the model has fairly good
discrimination (ability to separate those who did and did not experience
the event of interest).

\begin{Shaded}
\begin{Highlighting}[]
\NormalTok{pacman}\SpecialCharTok{::}\FunctionTok{p\_load}\NormalTok{(yardstick)}

\CommentTok{\# combine dataset}
\NormalTok{data\_grouped }\OtherTok{\textless{}{-}} \FunctionTok{bind\_rows}\NormalTok{(data\_train }\SpecialCharTok{\%\textgreater{}\%} \FunctionTok{mutate}\NormalTok{(}\AttributeTok{set =} \StringTok{"Training"}\NormalTok{),}
\NormalTok{                      data\_test }\SpecialCharTok{\%\textgreater{}\%} \FunctionTok{mutate}\NormalTok{(}\AttributeTok{set =} \StringTok{"Validation"}\NormalTok{)) }\SpecialCharTok{\%\textgreater{}\%} 
  \FunctionTok{group\_by}\NormalTok{(set)}

\CommentTok{\# Calculate ROC}
\NormalTok{data\_grouped }\SpecialCharTok{\%\textgreater{}\%} 
  \CommentTok{\# set the second level of factor variable as the event (i.e. dead)}
  \FunctionTok{roc\_auc}\NormalTok{(}\AttributeTok{truth =}\NormalTok{ dead, prob\_dead, }\AttributeTok{event\_level =} \StringTok{"second"}\NormalTok{)}
\CommentTok{\#\textgreater{} \# A tibble: 2 x 4}
\CommentTok{\#\textgreater{}   set        .metric .estimator .estimate}
\CommentTok{\#\textgreater{}   \textless{}chr\textgreater{}      \textless{}chr\textgreater{}   \textless{}chr\textgreater{}          \textless{}dbl\textgreater{}}
\CommentTok{\#\textgreater{} 1 Training   roc\_auc binary         0.767}
\CommentTok{\#\textgreater{} 2 Validation roc\_auc binary         0.783}

\CommentTok{\# Visualise ROC}
\NormalTok{data\_roc }\OtherTok{\textless{}{-}}\NormalTok{ data\_grouped }\SpecialCharTok{\%\textgreater{}\%} 
  \FunctionTok{roc\_curve}\NormalTok{(}\AttributeTok{truth =}\NormalTok{ dead, prob\_dead, }\AttributeTok{event\_level =} \StringTok{"second"}\NormalTok{)}

\FunctionTok{ggplot}\NormalTok{(data\_roc, }\FunctionTok{aes}\NormalTok{(}\AttributeTok{x =} \DecValTok{1}\SpecialCharTok{{-}}\NormalTok{specificity, }\AttributeTok{y =}\NormalTok{ sensitivity, }\AttributeTok{color =}\NormalTok{ set)) }\SpecialCharTok{+}
  \FunctionTok{geom\_abline}\NormalTok{(}\AttributeTok{slope =} \DecValTok{1}\NormalTok{, }\AttributeTok{intercept =} \DecValTok{0}\NormalTok{,}
              \AttributeTok{size =} \FloatTok{0.4}\NormalTok{, }\AttributeTok{color =} \StringTok{"grey21"}\NormalTok{, }\AttributeTok{linetype =} \StringTok{"dashed"}\NormalTok{) }\SpecialCharTok{+}
  \FunctionTok{geom\_line}\NormalTok{() }\SpecialCharTok{+}
  \FunctionTok{theme\_minimal}\NormalTok{(}\AttributeTok{base\_size =} \DecValTok{12}\NormalTok{) }\SpecialCharTok{+}
  \FunctionTok{coord\_equal}\NormalTok{()}
\CommentTok{\#\textgreater{} Warning: Using \textasciigrave{}size\textasciigrave{} aesthetic for lines was deprecated in ggplot2 3.4.0.}
\CommentTok{\#\textgreater{} i Please use \textasciigrave{}linewidth\textasciigrave{} instead.}
\CommentTok{\#\textgreater{} This warning is displayed once every 8 hours.}
\CommentTok{\#\textgreater{} Call \textasciigrave{}lifecycle::last\_lifecycle\_warnings()\textasciigrave{} to see where this warning was}
\CommentTok{\#\textgreater{} generated.}
\end{Highlighting}
\end{Shaded}

\includegraphics{risk_predict_files/figure-latex/roc-1.pdf}

\hypertarget{hosmer-lemeshow}{%
\paragraph{Hosmer-Lemeshow}\label{hosmer-lemeshow}}

The Hosmer-Lemeshow goodness of fit table for the two (S=0 and the S=1)
datasets were very similar. Both showed evidence of a well calibrated
model.

\begin{Shaded}
\begin{Highlighting}[]
\NormalTok{pacman}\SpecialCharTok{::}\FunctionTok{p\_load}\NormalTok{(generalhoslem)}

\NormalTok{gof }\OtherTok{\textless{}{-}}\NormalTok{ data\_grouped }\SpecialCharTok{\%\textgreater{}\%} 
  \FunctionTok{group\_map}\NormalTok{( }\SpecialCharTok{\textasciitilde{}} \FunctionTok{logitgof}\NormalTok{(}
    \AttributeTok{obs =}\NormalTok{ .}\SpecialCharTok{$}\NormalTok{dead,}
    \AttributeTok{exp =}\NormalTok{ .}\SpecialCharTok{$}\NormalTok{prob\_dead,}
    \AttributeTok{g =} \DecValTok{10}
\NormalTok{  ))}

\CommentTok{\# Fix name}
\FunctionTok{names}\NormalTok{(gof) }\OtherTok{\textless{}{-}} \FunctionTok{group\_keys}\NormalTok{(data\_grouped)[[}\FunctionTok{group\_vars}\NormalTok{(data\_grouped)]]}

\CommentTok{\# output Goodness of Fit metrics}
\NormalTok{gof}
\CommentTok{\#\textgreater{} $Training}
\CommentTok{\#\textgreater{} }
\CommentTok{\#\textgreater{}  Hosmer and Lemeshow test (binary model)}
\CommentTok{\#\textgreater{} }
\CommentTok{\#\textgreater{} data:  .$dead, .$prob\_dead}
\CommentTok{\#\textgreater{} X{-}squared = 8.105, df = 8, p{-}value = 0.4233}
\CommentTok{\#\textgreater{} }
\CommentTok{\#\textgreater{} }
\CommentTok{\#\textgreater{} $Validation}
\CommentTok{\#\textgreater{} }
\CommentTok{\#\textgreater{}  Hosmer and Lemeshow test (binary model)}
\CommentTok{\#\textgreater{} }
\CommentTok{\#\textgreater{} data:  .$dead, .$prob\_dead}
\CommentTok{\#\textgreater{} X{-}squared = 4.7631, df = 8, p{-}value = 0.7826}

\CommentTok{\# create GoF table}
\NormalTok{gof\_table }\OtherTok{\textless{}{-}} \FunctionTok{lapply}\NormalTok{(gof, }\ControlFlowTok{function}\NormalTok{(x)\{}
  \FunctionTok{cbind}\NormalTok{(x}\SpecialCharTok{$}\NormalTok{observed, x}\SpecialCharTok{$}\NormalTok{expected) }\SpecialCharTok{\%\textgreater{}\%}
    \FunctionTok{as\_tibble}\NormalTok{(}\AttributeTok{rownames =} \StringTok{"threshold"}\NormalTok{) }\SpecialCharTok{\%\textgreater{}\%} 
    \FunctionTok{mutate}\NormalTok{(}\AttributeTok{group =} \DecValTok{1}\SpecialCharTok{:}\FunctionTok{nrow}\NormalTok{(.))}
\NormalTok{\})}

\NormalTok{gof\_table}\SpecialCharTok{$}\NormalTok{Training}
\CommentTok{\#\textgreater{} \# A tibble: 10 x 6}
\CommentTok{\#\textgreater{}    threshold          y0    y1 yhat0 yhat1 group}
\CommentTok{\#\textgreater{}    \textless{}chr\textgreater{}           \textless{}dbl\textgreater{} \textless{}dbl\textgreater{} \textless{}dbl\textgreater{} \textless{}dbl\textgreater{} \textless{}int\textgreater{}}
\CommentTok{\#\textgreater{}  1 [0.0134,0.0541]    94     6  96.2  3.79     1}
\CommentTok{\#\textgreater{}  2 (0.0541,0.085]     94     6  92.9  7.09     2}
\CommentTok{\#\textgreater{}  3 (0.085,0.115]      92     8  90.0 10.0      3}
\CommentTok{\#\textgreater{}  4 (0.115,0.153]      86    14  86.8 13.2      4}
\CommentTok{\#\textgreater{}  5 (0.153,0.204]      77    23  82.4 17.6      5}
\CommentTok{\#\textgreater{}  6 (0.204,0.267]      84    16  76.3 23.7      6}
\CommentTok{\#\textgreater{}  7 (0.267,0.337]      68    32  69.8 30.2      7}
\CommentTok{\#\textgreater{}  8 (0.337,0.415]      65    35  62.9 37.1      8}
\CommentTok{\#\textgreater{}  9 (0.415,0.522]      50    50  53.4 46.6      9}
\CommentTok{\#\textgreater{} 10 (0.522,0.841]      39    61  38.2 61.8     10}
\NormalTok{gof\_table}\SpecialCharTok{$}\NormalTok{Validation}
\CommentTok{\#\textgreater{} \# A tibble: 10 x 6}
\CommentTok{\#\textgreater{}    threshold          y0    y1 yhat0 yhat1 group}
\CommentTok{\#\textgreater{}    \textless{}chr\textgreater{}           \textless{}dbl\textgreater{} \textless{}dbl\textgreater{} \textless{}dbl\textgreater{} \textless{}dbl\textgreater{} \textless{}int\textgreater{}}
\CommentTok{\#\textgreater{}  1 [0.0149,0.0582]    95     5  95.9  4.12     1}
\CommentTok{\#\textgreater{}  2 (0.0582,0.0897]    94     6  92.6  7.43     2}
\CommentTok{\#\textgreater{}  3 (0.0897,0.121]     90    10  89.5 10.5      3}
\CommentTok{\#\textgreater{}  4 (0.121,0.161]      88    12  86.0 14.0      4}
\CommentTok{\#\textgreater{}  5 (0.161,0.208]      84    16  81.6 18.4      5}
\CommentTok{\#\textgreater{}  6 (0.208,0.274]      74    26  76.2 23.8      6}
\CommentTok{\#\textgreater{}  7 (0.274,0.35]       75    25  69.0 31.0      7}
\CommentTok{\#\textgreater{}  8 (0.35,0.437]       62    38  60.3 39.7      8}
\CommentTok{\#\textgreater{}  9 (0.437,0.547]      50    50  50.9 49.1      9}
\CommentTok{\#\textgreater{} 10 (0.547,0.85]       30    70  35.8 64.2     10}
\end{Highlighting}
\end{Shaded}

Bar graphs comparing the predicted and observed risks in the S=0 and S=1
datasets also show good calibration (in both the training and validation
data).

\begin{Shaded}
\begin{Highlighting}[]
\CommentTok{\# tidy up gof\_table for plotting}
\NormalTok{tidy\_gof }\OtherTok{\textless{}{-}} \FunctionTok{bind\_rows}\NormalTok{(gof\_table, }\AttributeTok{.id =} \StringTok{"set"}\NormalTok{) }\SpecialCharTok{\%\textgreater{}\%}
\NormalTok{  dplyr}\SpecialCharTok{::}\FunctionTok{select}\NormalTok{(set, group, }\AttributeTok{Obs =}\NormalTok{ y1, }\AttributeTok{Exp =}\NormalTok{ yhat1) }\SpecialCharTok{\%\textgreater{}\%} 
  \FunctionTok{pivot\_longer}\NormalTok{(}
    \AttributeTok{cols =} \FunctionTok{c}\NormalTok{(Obs, Exp),}
    \AttributeTok{names\_to =} \StringTok{"class"}
\NormalTok{  )}

\CommentTok{\# plotting}
\NormalTok{tidy\_gof }\SpecialCharTok{\%\textgreater{}\%} 
  \FunctionTok{ggplot}\NormalTok{(}\FunctionTok{aes}\NormalTok{(}\AttributeTok{x =}\NormalTok{ group, }\AttributeTok{y =}\NormalTok{ value, }\AttributeTok{fill =}\NormalTok{ class)) }\SpecialCharTok{+}
  \FunctionTok{facet\_grid}\NormalTok{(}\AttributeTok{cols =} \FunctionTok{vars}\NormalTok{(set)) }\SpecialCharTok{+}
  \FunctionTok{geom\_bar}\NormalTok{(}\AttributeTok{stat =} \StringTok{"identity"}\NormalTok{, }\AttributeTok{position =} \FunctionTok{position\_dodge}\NormalTok{()) }\SpecialCharTok{+}
  \FunctionTok{scale\_x\_continuous}\NormalTok{(}\AttributeTok{breaks =} \DecValTok{1}\SpecialCharTok{:}\DecValTok{10}\NormalTok{) }\SpecialCharTok{+}
  \FunctionTok{theme\_minimal}\NormalTok{() }\SpecialCharTok{+}
  \FunctionTok{theme}\NormalTok{(}\AttributeTok{panel.grid.minor.x =} \FunctionTok{element\_blank}\NormalTok{(),}
        \AttributeTok{panel.grid.major.x =} \FunctionTok{element\_blank}\NormalTok{(),}
        \AttributeTok{axis.title =} \FunctionTok{element\_blank}\NormalTok{())}
\end{Highlighting}
\end{Shaded}

\includegraphics{risk_predict_files/figure-latex/gof_bar-1.pdf}

\hypertarget{calibration-curve}{%
\subsubsection{Calibration curve}\label{calibration-curve}}

\begin{Shaded}
\begin{Highlighting}[]
\NormalTok{df\_model }\OtherTok{\textless{}{-}}\NormalTok{ model }\SpecialCharTok{\%\textgreater{}\%}
  \CommentTok{\# augment creates new columns with some useful information from the model}
  \CommentTok{\# .fitted = predicted values}
  \FunctionTok{augment}\NormalTok{(}\AttributeTok{type.predict =} \StringTok{"link"}\NormalTok{, }\AttributeTok{newdata =}\NormalTok{ data\_test) }\SpecialCharTok{\%\textgreater{}\%}  
  \FunctionTok{mutate}\NormalTok{(}\AttributeTok{dead\_count =} \FunctionTok{as.numeric}\NormalTok{(dead) }\SpecialCharTok{{-}} \DecValTok{1}\NormalTok{) }\SpecialCharTok{\%\textgreater{}\%} 
  \FunctionTok{arrange}\NormalTok{(.fitted)}

\CommentTok{\# calculate alpha (calibration at large) \& beta (calibration slope)}
\CommentTok{\# alpha is calculated by constraining beta to 1}

\NormalTok{df\_alpha }\OtherTok{\textless{}{-}} \FunctionTok{glm}\NormalTok{(dead\_count }\SpecialCharTok{\textasciitilde{}} \FunctionTok{offset}\NormalTok{(.fitted),}
               \AttributeTok{data =}\NormalTok{ df\_model,}
               \AttributeTok{family =}\NormalTok{ binomial) }\SpecialCharTok{\%\textgreater{}\%} 
  \FunctionTok{tidy}\NormalTok{(}\AttributeTok{conf.int =} \ConstantTok{TRUE}\NormalTok{)}

\NormalTok{alpha\_text }\OtherTok{\textless{}{-}} \FunctionTok{paste0}\NormalTok{(}\FunctionTok{round}\NormalTok{(df\_alpha}\SpecialCharTok{$}\NormalTok{estimate[}\DecValTok{1}\NormalTok{], }\DecValTok{2}\NormalTok{),}
                     \StringTok{" (95\% CI: "}\NormalTok{, }\FunctionTok{round}\NormalTok{(df\_alpha}\SpecialCharTok{$}\NormalTok{conf.low[}\DecValTok{1}\NormalTok{], }\DecValTok{2}\NormalTok{),}
                     \StringTok{" to "}\NormalTok{, }\FunctionTok{round}\NormalTok{(df\_alpha}\SpecialCharTok{$}\NormalTok{conf.high[}\DecValTok{1}\NormalTok{], }\DecValTok{2}\NormalTok{),}
                     \StringTok{")"}\NormalTok{)}

\CommentTok{\# beta is calculated by constraining alpha to 0}
\NormalTok{df\_beta }\OtherTok{\textless{}{-}} \FunctionTok{glm}\NormalTok{(dead\_count }\SpecialCharTok{\textasciitilde{}}\NormalTok{ .fitted }\SpecialCharTok{{-}} \DecValTok{1}\NormalTok{,}
                \AttributeTok{data =}\NormalTok{ df\_model,}
                \AttributeTok{family =}\NormalTok{ binomial) }\SpecialCharTok{\%\textgreater{}\%} 
  \FunctionTok{tidy}\NormalTok{(}\AttributeTok{conf.int =} \ConstantTok{TRUE}\NormalTok{)}

\NormalTok{beta\_text }\OtherTok{\textless{}{-}} \FunctionTok{paste0}\NormalTok{(}\FunctionTok{round}\NormalTok{(df\_beta}\SpecialCharTok{$}\NormalTok{estimate[}\DecValTok{1}\NormalTok{], }\DecValTok{2}\NormalTok{),}
                    \StringTok{" (95\% CI: "}\NormalTok{, }\FunctionTok{round}\NormalTok{(df\_beta}\SpecialCharTok{$}\NormalTok{conf.low[}\DecValTok{1}\NormalTok{], }\DecValTok{2}\NormalTok{),}
                    \StringTok{" to "}\NormalTok{, }\FunctionTok{round}\NormalTok{(df\_beta}\SpecialCharTok{$}\NormalTok{conf.high[}\DecValTok{1}\NormalTok{], }\DecValTok{2}\NormalTok{),}
                    \StringTok{")"}\NormalTok{)}


\CommentTok{\# calculate mean predicted results per decile}
\NormalTok{df\_sum\_decile }\OtherTok{\textless{}{-}}\NormalTok{ model }\SpecialCharTok{\%\textgreater{}\%}
  \CommentTok{\# augment creates new columns with some useful information from the model}
  \CommentTok{\# .fitted = predicted values}
  \FunctionTok{augment}\NormalTok{(}\AttributeTok{type.predict =} \StringTok{"response"}\NormalTok{, }\AttributeTok{newdata =}\NormalTok{ data\_test) }\SpecialCharTok{\%\textgreater{}\%}  
  \FunctionTok{mutate}\NormalTok{(}\AttributeTok{dead\_count =} \FunctionTok{as.numeric}\NormalTok{(dead) }\SpecialCharTok{{-}} \DecValTok{1}\NormalTok{) }\SpecialCharTok{\%\textgreater{}\%} 
  \FunctionTok{arrange}\NormalTok{(.fitted) }\SpecialCharTok{\%\textgreater{}\%} 
  \FunctionTok{mutate}\NormalTok{(}\AttributeTok{decile =} \FunctionTok{ntile}\NormalTok{(.fitted, }\DecValTok{10}\NormalTok{)) }\SpecialCharTok{\%\textgreater{}\%} 
  \FunctionTok{group\_by}\NormalTok{(decile) }\SpecialCharTok{\%\textgreater{}\%} 
  \FunctionTok{summarize}\NormalTok{(}\AttributeTok{mean\_prediction =} \FunctionTok{mean}\NormalTok{(.fitted), }
            \AttributeTok{sd\_prediction =} \FunctionTok{sd}\NormalTok{(.fitted),}
            \AttributeTok{prop\_observed =} \FunctionTok{mean}\NormalTok{(dead\_count)) }\SpecialCharTok{\%\textgreater{}\%} 
  \FunctionTok{mutate}\NormalTok{(}\AttributeTok{dummy =} \StringTok{"decile"}\NormalTok{)}

\CommentTok{\# plotting}
\FunctionTok{ggplot}\NormalTok{(df\_sum\_decile, }\FunctionTok{aes}\NormalTok{(}\AttributeTok{x =}\NormalTok{ mean\_prediction, }\AttributeTok{y =}\NormalTok{ prop\_observed)) }\SpecialCharTok{+}
  \FunctionTok{theme\_minimal}\NormalTok{() }\SpecialCharTok{+}
  \FunctionTok{geom\_abline}\NormalTok{(}\AttributeTok{slope =} \DecValTok{1}\NormalTok{, }\AttributeTok{intercept =} \DecValTok{0}\NormalTok{, }\AttributeTok{linetype =} \StringTok{"dashed"}\NormalTok{) }\SpecialCharTok{+}
  \CommentTok{\# geom\_abline(slope = 1, intercept = df\_alpha$estimate[1], color = "\#1b5299") +}
  \FunctionTok{geom\_point}\NormalTok{(}\FunctionTok{aes}\NormalTok{(}\AttributeTok{color =}\NormalTok{ dummy), }\AttributeTok{size =} \DecValTok{3}\NormalTok{) }\SpecialCharTok{+}
  \FunctionTok{scale\_color\_manual}\NormalTok{(}\AttributeTok{values =} \StringTok{"steelblue"}\NormalTok{) }\SpecialCharTok{+}
  \FunctionTok{labs}\NormalTok{(}\AttributeTok{x =} \StringTok{"Mean Predicted"}\NormalTok{, }\AttributeTok{y =} \StringTok{"Mean Observed"}\NormalTok{, }\AttributeTok{title =} \StringTok{"Calibration plot"}\NormalTok{) }\SpecialCharTok{+}
  \FunctionTok{annotate}\NormalTok{(}\StringTok{"label"}\NormalTok{, }\AttributeTok{x =} \SpecialCharTok{{-}}\ConstantTok{Inf}\NormalTok{, }\AttributeTok{y =} \ConstantTok{Inf}\NormalTok{, }\AttributeTok{hjust =} \DecValTok{0}\NormalTok{, }\AttributeTok{vjust =} \FloatTok{1.5}\NormalTok{, }\AttributeTok{label.size =} \DecValTok{0}\NormalTok{,}
           \AttributeTok{label =} \FunctionTok{paste}\NormalTok{(}\StringTok{"calibration{-}in{-}the{-}large ="}\NormalTok{,}
\NormalTok{                         alpha\_text)) }\SpecialCharTok{+}
  \FunctionTok{annotate}\NormalTok{(}\StringTok{"label"}\NormalTok{, }\AttributeTok{x =} \SpecialCharTok{{-}}\ConstantTok{Inf}\NormalTok{, }\AttributeTok{y =} \ConstantTok{Inf}\NormalTok{, }\AttributeTok{hjust =} \DecValTok{0}\NormalTok{, }\AttributeTok{vjust =} \FloatTok{2.5}\NormalTok{, }\AttributeTok{label.size =} \DecValTok{0}\NormalTok{,}
           \AttributeTok{label =} \FunctionTok{paste}\NormalTok{(}\StringTok{"calibration slope ="}\NormalTok{,}
\NormalTok{                         beta\_text)) }\SpecialCharTok{+}
  \FunctionTok{theme}\NormalTok{(}\AttributeTok{legend.title =} \FunctionTok{element\_blank}\NormalTok{())}
\end{Highlighting}
\end{Shaded}

\includegraphics{risk_predict_files/figure-latex/unnamed-chunk-2-1.pdf}

Since calibration-in-the-large is close to 0, and calibration slope is
close to 1, we can say that the model is well calibrated

\end{document}
